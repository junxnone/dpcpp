本书出版时,暂定的SYCL 2020规范可供公众评论。随着时间的推移,目前的SYCL 1.2.1标准会有一个后续版本。预计后期会称为SYCL 2020。虽然很高兴地说,这本书教授的是SYCL 2020,但这个标准目前还不存在。\par

本书介绍了SYCL的扩展,以大致了解SYCL将来的发展方向。这些扩展在DPC++编译器项目中都有实现。几乎在DPC++中实现的扩展都是暂定为SYCL 2020规范中的新特性。DPC++支持的新特性包括USM、子组、C++17支持的语法简化(称为CTAD——class模板参数推导),以及无需命名就可以使用匿名lambdas函数。\par

发布时,SYCL编译器(包括DPC++)没有实现SYCL 2020临时规范中的功能。\par

本书中使用的一些特性是DPC++编译器特有的。这些特性中有许多是Intel对SYCL的扩展,后来被SYCL 2020临时规范所接受,在一些情况下,其语法在标准化过程中发生了细微的变化。其他特性仍在开发或讨论中,以可能会包含在未来的SYCL标准中,其语法也可能也会修改。语言开发过程中,这样的语法变化实际上非常需要,因为我们希望特性能够进化和改进,从而满足更广泛的开发人员群体和更广泛设备的功能需求。本书中的所有代码示例都使用了DPC++语法,以确保与DPC++编译器的兼容性。\par

在努力接近SYCL的发展方向的同时,需要对本书中的信息进行调整,以便与标准的发展保持一致。更新信息的重要资源包括本书GitHub和勘误表,可以从本书的网页(www.apress.com/9781484255735)找到,以及在线oneAPI DPC++语言参考(tinyurl.com/dpcppref)。\par

















