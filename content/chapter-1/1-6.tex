第2章将讨论队列和操作,但是我们可以从一个简单的开始了解。队列是允许应用程序在设备上完成工作的方式。有两种类型的操作可以放在队列中:(a)要执行的代码和(b)内存操作。执行代码通过\textit{single\_task}、\textit{parallel\_for}或\textit{parallel\_for\_work\_group}表示。内存操作用来对主机与设备之间的复制或填充,从而初始化内存。只有需要做的更多的控制时,才需要使用内存操作。这些都会从本书第2章开始讨论。现在,我们知道了到队列是连接命令与设备的桥梁,我们有一组操作可以放入队列来执行代码和/或移动数据。同样重要的是,要理解请求的操作放置在队列中而不需要等待。在将操作提交到队列后,主机继续执行程序,而设备最终将以异步执行的方式执行队列请求的操作。\par

\begin{tcolorbox}[colback=red!5!white,colframe=red!75!black]
队列将连接到设备。\par
我们将操作提交到这些队列中,从而进行计算或数据移动。\par
行为是异步发生的。\par
\end{tcolorbox}