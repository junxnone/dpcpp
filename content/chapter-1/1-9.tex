术语“并行”和“并行”并不等同,尽管它们有时会误解为等同的。重要的是,并发所需要的编程考虑对并行性也很重要。\par

术语“并发”指的是可以启动的代码,但不一定是在同一时刻。在我们的计算机上,如果我们打开了一个邮件程序和一个Web浏览器,那么它们是同时运行的。在只有一个处理器的系统上,通过一个时间切片(在运行每个程序之间快速来回切换)的过程可以发生并发。\par

\begin{tcolorbox}[colback=blue!5!white,colframe=blue!75!black]
\textbf{Tip} \\
对并发性的任何考虑,对并行性也很重要。
\end{tcolorbox}

术语“并行”指的是可以在同一时刻启动的代码。并行要求系统实际上可以同时做不止一件事。一个异构的系统总是可以并行地做事情,因为它的本质是至少有两个计算设备。当然,SYCL程序不需要异构系统,因为它可以仅在主机系统上运行。现今,任何主机系统都可以并行执行。\par

代码的并发执行通常面临着与代码并行执行相同的问题,因为任何特定的代码都不能假设它是唯一的(数据位置、I/O等)。\par








