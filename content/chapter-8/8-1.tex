在第3章中,我们讨论了数据管理和数据使用的排序。那一章描述了DPC++中图背后的关键抽象:依赖。内核之间的依赖基本上是基于内核访问的数据。在计算输出之前,内核需要确定是否读取了正确的数据。\par

我们描述了三种类型的数据依赖,它们对于确保正确执行非常重要。第一种是读后写(RAW),当一个任务需要读取由另一个任务产生的数据时发生。这种类型的依赖描述了两个内核之间的数据流。第二种依赖发生在一个任务在另一个任务读取数据后需要更新数据时。我们称之为写后读(WAR)依赖。当两个任务试图写入相同的数据时,就会出现最后一种数据依赖。这就是写后写(WAW)依赖关系。\par

数据依赖关系是用于构建图表的构建块。这组依赖关系是我们所需要的,既可以表示简单的线性链,也可以表示具有复杂依赖关系的大型复杂图。无论计算需要哪种类型的图,DPC++图都能确保程序基于依赖正确执行。然而,开发者必须确保图能正确地表达程序中的所有依赖关系。\par