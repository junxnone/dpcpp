USM是可供使用的数据管理工具。当移植大量使用指针的工程时,USM可以简化工作。支持USM的设备支持统一虚拟地址空间,拥有统一虚拟地址空间意味着主机上的USM返回的指针都是设备上的有效指针。不需要手动转换主机指针来获得“设备指针”——可以在主机和设备上看到相同的指针。\par

关于USM更详细的解读会在第6章继续。\par

\hspace*{\fill} \par %插入空行
\textbf{通过指针访问内存}

当系统包含主机内存和设备内存时,不是所有的内存都相同,所以USM定义了三种不同的分配类型:设备、主机和共享。所有类型的分配都在主机上执行。图3-3总结了各分配类型的特点。\par

\hspace*{\fill} \par %插入空行
图3-3 USM分配类型
\begin{table}[H]
	\begin{tabular}{|l|l|l|l|l|}
		\hline
		\textbf{分配类型} & \textbf{描述}                       & \textbf{可访问主机?} & \textbf{可访问设备?} & \textbf{位于}        \\ \hline
		\textbf{设备}          & 设备内存的分配               & \XSolidBrush                       & \Checkmark                              & 设备                     \\ \hline
		\textbf{主机}            & 主机内存的分配                & \Checkmark                            & \Checkmark                               & 主机                       \\ \hline
		\textbf{共享}          & 主机和设备共享 & \Checkmark                            & \Checkmark                               & 随意迁移 \\ \hline
	\end{tabular}
\end{table}

设备分配发生在设备内存中,分配的内存可以从设备上读取和写入,但不能直接从主机上访问。必须使用显式的复制操作,在主机内存和设备内存之间移动数据。\par

主机和设备上都可以访问主机内存,这意味着相同的指针在主机代码和设备内核中都有效。然而,访问这样的指针时,数据总是来自主机内存。当访问设备时,数据不会从主机迁移到设备内存。相反,数据通常通过总线发送,例如PCI-Express (PCI-E)将设备连接到主机。\par

共享分配的内存在主机和设备上都可以访问,非常类似于主机分配,不同之处在于数据可以在主机内存和设备本地内存之间迁移。迁移之后,对设备的访问将在设备内存中进行,而不是远程访问主机内存。通常,这是通过运行时内部的机制和底层驱动实现。\par

\hspace*{\fill} \par %插入空行
\textbf{USM和数据移动}

USM支持显式和隐式的数据移动策略,不同的分配类型对应不同的策略。设备内存要求显式地在主机和设备之间移动数据,而主机和共享内存提供隐式的数据移动。\par

\hspace*{\fill} \par %插入空行
\textbf{USM显式数据移动}

使用USM的显式数据移动,通过设备内存和在队列和处理程序中使用特殊的memcpy()完成的。将memcpy()操作(动作)放入队列,将数据从主机传输到设备,或从设备传输到主机。\par

图3-4包含操作设备分配的内核。内核执行前后使用memcpy()操作,hostArray和deviceArray之间复制数据。队列上调用wait(),确保在内核执行之前复制到设备的操作已经完成,并确保数据复制回主机之前内核已经完成。我们将在本章后面学习如何消除这些调用。\par

\hspace*{\fill} \par %插入空行
图3-4 显式地数据移动
\begin{lstlisting}[caption={}]
#include <CL/sycl.hpp>
#include<array>
using namespace sycl;
constexpr int N = 42; 

int main() {
	queue Q;
	
	std::array<int,N> host_array;
	int *device_array = malloc_device<int>(N, Q);
	
	for (int i = 0; i < N; i++)
		host_array[i] = N;
		
	// We will learn how to simplify this example later
	Q.submit([&](handler &h) {
		// copy hostArray to deviceArray
		h.memcpy(device_array, &host_array[0], N * sizeof(int));
	});
	Q.wait();
	
	
	Q.submit([&](handler &h) {
		h.parallel_for(N, [=](id<1> i) { device_array[i]++; }); 
	});
	Q.wait();
	
	Q.submit([&](handler &h) {
		// copy deviceArray back to hostArray
		h.memcpy(&host_array[0], device_array, N * sizeof(int)); 
	});
	Q.wait();
	
	free(device_array, Q);
	return 0;
}
\end{lstlisting}

\hspace*{\fill} \par %插入空行
\textbf{USM隐式数据移动}

使用USM的隐式数据移动是通过主机和共享内存完成的。使用这类型的内存,不需要显式地插入复制操作。相反,只需访问内核中的指针,任何数据移动都可以自动执行,无需手动干预(只要设备支持这些内存分配)。这提高了代码的移植性:只需用适当的USM分配函数,替换malloc或new(以及free),一切就好了。\par

\hspace*{\fill} \par %插入空行
图3-5 USM隐式地数据移动
\begin{lstlisting}[caption={}]
#include <CL/sycl.hpp>
using namespace sycl;
constexpr int N = 42;

int main() {
	queue Q;
	int *host_array = malloc_host<int>(N, Q);
	int *shared_array = malloc_shared<int>(N, Q);
	
	for (int i = 0; i < N; i++) {
		// Initialize hostArray on host
		host_array[i] = i;
	}

	// We will learn how to simplify this example later
	Q.submit([&](handler &h) {
		h.parallel_for(N, [=](id<1> i) {
			// access sharedArray and hostArray on device
			shared_array[i] = host_array[i] + 1;
		});
	});
	Q.wait();
	
	for (int i = 0; i < N; i++) {
		// access sharedArray on host
		host_array[i] = shared_array[i];
	}

	free(shared_array, Q);
	free(host_array, Q);
	return 0;
}
\end{lstlisting}

图3-5中,创建了两个数组,hostArray和sharedArray,分别是主机内存和共享内存。虽然主机和共享内存都可以在主机代码中访问,但这里只初始化了hostArray。类似地,可以在内核直接访问,执行远程读取数据。运行时确保sharedArray在内核访问之前,数据在设备上是可用的,并且在之后主机代码读取时会回移数据,这些都不需要开发者干预。\par




















