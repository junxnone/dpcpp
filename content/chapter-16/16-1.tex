DPC++为并行化应用程序,或从头开始开发并行应用程序铺了一条道路。当一个程序在CPU上运行时,其性能在很大程度上取决于以下因素:\par

\begin{itemize}
	\item 内核代码的单个调用和执行的底层性能
	\item 在并行内核中运行的程序的百分比及其可扩展性
	\item CPU利用率、有效的数据共享、数据局部性和负载均衡
	\item 工作项之间同步化和通信的数量
	\item 为创建、恢复、管理、挂起、销毁和同步工作项所执行的线程而引入的开销,串行到并行或并行到串行转换的数量会使性能变得更糟
	\item 由共享内存或错误的共享内存引起的内存冲突
	\item 共享资源(如内存、写入组合缓冲区和内存带宽)的性能限制
\end{itemize}

此外,与任何处理器类型一样,CPU可能因厂商的不同而不同,甚至因产品的不同而不同。对于一个CPU的最佳实践,可能不是针对其他CPU和配置的最佳实践。\par

\begin{tcolorbox}[colback=red!5!white,colframe=red!75!black]
要在CPU上实现最佳性能,请尽可能多地了解CPU架构的特征!
\end{tcolorbox}











