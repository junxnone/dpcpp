前面的部分中,内核是API定义的,或者是由特定的句柄创建的,内核通过两个步骤创建:首先创建程序对象,然后从程序对象创建内核。程序对象是内核及其调用的函数集合,这些函数编译为一个单元。\par

对于以Lambda或命名函数对象表示的内核,包含内核的程序对象是隐式的,对应用程序来说不可见。对于需要更多控制的应用程序,可以显式地管理内核和封装程序对象。为了说明为什么这样做,简要地了解下有多少SYCL实现管理即时(JIT)内核编译就能知道了。\par

虽然规范并不要求,但许多实现使用“惰性的”方式编译内核。这是一个很好的策略,因为可以确保快速启动程序,并且仅编译执行的内核。这种策略的缺点是,第一次使用内核的时间通常比后续要长,因为包括编译所需的时间,以及提交和执行内核的时间。对于某些复杂的内核,编译内核的时间可能很长,因此需要在应用程序执行期间将编译进行转移,比如:在应用程序加载时,或者在后台线程中。\par

一些内核还可能受益于实现定义的“构建选项”,以精确地控制内核的编译方式。例如:可以指示内核编译器使用精度较低,但性能更好的数学库。\par

为了更好地控制编译内核的时间和方式,应用程序可以使用特定的构建选项,在使用内核之前显式地编译内核。然后,将预编译的内核提交到一个队列中执行。原理如图10-9所示。\par

\hspace*{\fill} \par %插入空行
图10-9 使用构建选项编译内核Lambda函数
\begin{lstlisting}[caption={}]
// This compiles the kernel named by the specified template
// parameter using the "fast relaxed math" build option.
program p(Q.get_context());

p.build_with_kernel_type<class Add>("-cl-fast-relaxed-math");

Q.submit([&](handler& h) {
	accessor data_acc {data_buf, h};
	
	h.parallel_for<class Add>(
		// This uses the previously compiled kernel.
		p.get_kernel<class Add>(),
		range{size},
		[=](id<1> i) {
			data_acc[i] = data_acc[i] + 1;
	});
});
\end{lstlisting}

本例中,从SYCL上下文创建程序对象,使用build\_with\_kernel\_type函数构建由指定的模板形参定义的内核。对于这个示例,程序构建选项-cl-fast-relaxed-math表示内核编译器可以使用更快的数学库,但程序构建选项是可选的,如果不需要特殊的程序构建选项,可以省略。本例中,需要指定内核Lambda的模板参数,以确定要编译的内核。\par

也可以从特定设备的上下文上创建程序对象,程序对象可以使用不同的构建选项,将内核编译到不同的设备对象上。\par

除了内核Lambda外,之前编译的内核使用get\_kernel将内核传递给parallel\_for。这可以确保使用使用高效数学库构建的内核。如果之前编译的内核没有传递给parallel\_for,那么内核将再次编译,不需要任何构建选项。这可能在功能上是正确的,但肯定不是预期的行为!\par

许多情况下,这些步骤不太可能对应用程序的行为产生影响,但是在调优应用程序的性能时,需要考虑这些步骤所带来的性能影响。\par

\begin{tcolorbox}[colback=blue!5!white,colframe=blue!75!black, title=改进互动性和程序对象管理]
虽然本章介绍了描述的用于互动性和程序对象管理的SYCL接口,但它们可能会在SYCL和DPC++的未来版本中得到改进和增强。请参考最新的SYCL和DPC++文档,以找到更新。
\end{tcolorbox}






































