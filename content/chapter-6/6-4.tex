现在我们了解了如何使用USM分配内存,我们将讨论如何管理数据。我们可以将其分为两部分:数据初始化和数据移动。\par

\hspace*{\fill} \par %插入空行
\textbf{初始化}

数据初始化关注的是对内存执行计算之前内存的填充值,常见初始化模式是在使用之前用零填充。如果要使用USM分配来实现这一点,可以通过多种方式来实现。首先,可以写一个内核函数来做这个。如果数据集特别大,或者初始化需要复杂的计算,这是合理的方法。其次,我们可以将其实现为遍历分配的所有元素的循环,将每个元素设置为0。然而,这种方法存在一个问题。循环可以很好地用于主机和共享分配,因为它们可以在主机上访问。但在主机上不能访问设备分配,主机代码中的循环将不能对进行写入。这就引出了第三种选择。\par

memset函数设计用来实现这个初始化模式。USM提供了一个memset,它是处理程序和队列类的成员函数。它有三个参数:表示要设置的内存基址的指针,表示要设置的模式的字节值,以及要设置为该模式的字节数。与主机上的循环不同,memset是并行发生的,并且与设备分配一起工作。\par

虽然memset是一个有用的操作,但只允许指定一个模式来填充分配。USM提供了fill方法(作为处理程序和队列类的成员),允许我们用任意模式填充内存。fill是一个函数,模板化在我们想要写入分配的模式的类型上。给它创建一个int型模板,然后用数字“42”填充分配。与memset类似,fill接受三个参数:指向要填充的分配的基址的指针,要填充的值,以及希望将该值写入分配的数量。\par

\hspace*{\fill} \par %插入空行
\textbf{数据移动}

数据移动可能是USM需要理解的重点。如果正确的数据没有在正确的时间出现在正确的地点,程序将产生不正确的结果。USM定义了两种用来管理数据的策略:显式和隐式。使用策略的选择与硬件支持,或使用的USM分配的类型有关。\par

\hspace*{\fill} \par %插入空行
\textbf{显式}

USM提供的第一个策略是显式数据移动(图6-6)。这里,必须显式地在主机和设备之间复制数据。可以通过调用memcpy方法来完成,该方法可以在处理程序和队列类上找到。memcpy方法有三个参数:指向目标内存的指针,指向源内存的指针,以及要在主机和设备之间复制的字节数。不需要指定复制将要发生的方向——这在源指针和目标指针中是隐式确定的。\par

显式数据移动最常见的用法是使用USM对设备分配内存中的数据进行复制,因为设备端内存在主机上不可访问。此外,这可能会造成错误:副本可能会忽略,不正确的数据量可能被复制,或者源或目标指针可能是不正确的。\par

然而,显式数据移动也有优点。给了我们很大的优势:完全控制数据移动。某些应用程序中,控制复制数据的数量和复制数据的时间对于获得最佳性能非常重要。理想情况下,可以将计算与数据移动重叠,确保硬件以高利用率运行。\par

其他类型的USM分配,不论是host,还是shared,都可以在主机和设备端访问,不需要显式地复制到设备。这就引出了USM中数据移动的另一种策略。\par

\hspace*{\fill} \par %插入空行
图6-6 USM显式移动数据
\begin{lstlisting}[caption={}]
constexpr int N = 42;

queue Q;

std::array<int,N> host_array;
int *device_array = malloc_device<int>(N, Q);
for (int i = 0; i < N; i++)
	host_array[i] = N;

Q.submit([&](handler& h) {
	// copy hostArray to deviceArray
	h.memcpy(device_array, &host_array[0], N * sizeof(int));
});

Q.wait(); // needed for now (we learn a better way later)

Q.submit([&](handler& h) {
	h.parallel_for(N, [=](id<1> i) {
		device_array[i]++;
	});
});

Q.wait(); // needed for now (we learn a better way later)

Q.submit([&](handler& h) {
	// copy deviceArray back to hostArray
	h.memcpy(&host_array[0], device_array, N * sizeof(int));
});

Q.wait(); // needed for now (we learn a better way later)

free(device_array, Q);
\end{lstlisting}

\hspace*{\fill} \par %插入空行
\textbf{隐式}

USM提供的第二种策略是隐式数据移动(示例用法如图6-7所示)在这个策略中,数据移动是隐式发生的。使用隐式数据移动,不需要插入memcpy,因为可以通过USM指针直接访问数据。而系统的任务是确保数据在使用时,在正确的位置上可用。\par

对于主机分配,人们可能会争论是否真的进行了数据移动。根据定义,它们始终指向主机内存,因此给定的主机指针表示的内存不能存储在设备上。但当在设备上访问主机分配时,数据移动就会发生。不是将内存迁移到设备,而是通过适当的接口将读或写的值传输到内核中。这对于数据不需要驻留在设备上的流内核很有用。\par

隐式数据移动主要与USM共享分配有关。这种类型的分配可以在主机和设备上访问,并且可以在主机和设备之间迁移。这种迁移是自动进行的,或者是隐式地进行的,只需访问不同位置的数据即可。接下来,我们将讨论在为共享分配进行数据迁移时需要考虑的几个问题。\par

\hspace*{\fill} \par %插入空行
图6-7 USM隐式数据移动
\begin{lstlisting}[caption={}]
constexpr int N = 42;

queue Q;

int* host_array = malloc_host<int>(N, Q);
int* shared_array = malloc_shared<int>(N, Q);
for (int i = 0; i < N; i++)
	host_array[i] = i;
	
Q.submit([&](handler& h) {
	h.parallel_for(N, [=](id<1> i) {
		// access sharedArray and hostArray on device
		shared_array[i] = host_array[i] + 1;
	});
});

Q.wait();

free(shared_array, Q);
free(host_array, Q);
\end{lstlisting}

\hspace*{\fill} \par %插入空行
\textbf{迁移}

通过显式数据移动,可以控制发生多少数据移动。使用隐式数据移动,系统可处理这一问题,但可能没有那么高效。DPC++运行时不是oracle——它不能预测应用将访问什么数据。此外,指针分析对于编译器来说仍然非常困难,它可能无法准确地分析和识别内核中可能使用的每个分配。因此,隐式数据移动机制的实现,可能会根据支持USM设备的功能做出不同的决定,这既影响共享分配的使用方式,也影响了它们的执行方式。\par

如果设备非常有能力,它可能能够根据需要迁移内存。这种情况下,数据移动将在主机或设备试图访问当前不在所需位置的分配之后发生。按需获取数据极大地简化了编程,因为它提供了语义,可以在任何地方访问USM共享指针,并且可以正常工作。如果设备不支持按需迁移(第12章解释了如何查询设备的功能),仍然能够保证相同的语义,并对共享指针的使用方式进行限制。\par

限制形式的USM共享分配管理着何时何地可以访问共享指针,以及共享分配的大小。如果设备不能按需迁移内存,则运行时必须保守,并假定内核可以访问其设备附加内存中的任何分配。这会带来两种后果。\par

首先,这意味着主机和设备不应该同时尝试访问共享分配。程序应该以阶段替代访问。主机可以访问分配,然后内核可以使用该数据进行计算,最后主机读取结果。\par

如果没有这个限制,主机可以访问内核的不同分配。这种并发访问通常发生在设备内存页上。主机可以访问一个页,而设备可以访问另一个页。第19章将介绍原子访问相同的数据块。\par

这种受限的共享分配形式的第二个后果是,分配受到设备内存总量的限制。如果设备不能按需迁移内存,则无法将数据迁移到主机,以便为不同的数据腾出空间。如果设备支持按需迁移,则可能会超量使用内存,从而允许内核计算超过设备内存通常包含的数据,而这种灵活性可能会因为数据移动,从而产生性能损失。\par

\hspace*{\fill} \par %插入空行
\textbf{细粒度控制}

当设备支持按需迁移共享分配时,当访问内存位置上没有相应数据时,需要进行数据移动。这时,内核在等待数据移动完成时可能会停止。接下来执行的语句甚至会产生更多的数据移动,并给内核执行带来更多的延迟。\par

DPC++为我们提供了一种修改自动迁移机制性能的方法。通过定义两个函数来做到这一点:prefetch和mem\_advise。图6-8展示了每种方法,这些函数让向运行时提供了内核函数如何访问数据的提示,以便运行时选择在内核访问数据之前开始移动数据。请注意,这个例子使用了直接在队列对象上调用parallel\_for的队列快捷方法,而不是在传递给submit(命令组)一个lambda调用。\par

\hspace*{\fill} \par %插入空行
图6-8 通过prefetch和mem\_advise进行细粒度控制
\begin{lstlisting}[caption={}]
// Appropriate values depend on your HW
constexpr int BLOCK_SIZE = 42;
constexpr int NUM_BLOCKS = 2500;
constexpr int N = NUM_BLOCKS * BLOCK_SIZE;

queue Q;
int *data = malloc_shared<int>(N, Q);
int *read_only_data = malloc_shared<int>(BLOCK_SIZE, Q);

// Never updated after initialization
for (int i = 0; i < BLOCK_SIZE; i++)
	read_only_data[i] = i;
	
// Mark this data as "read only" so the runtime can copy it
// to the device instead of migrating it from the host.
// Real values will be documented by your DPC++ backend.
int HW_SPECIFIC_ADVICE_RO = 0;

Q.mem_advise(read_only_data, BLOCK_SIZE, HW_SPECIFIC_ADVICE_RO);

event e = Q.prefetch(data, BLOCK_SIZE);

for (int b = 0; b < NUM_BLOCKS; b++) {
	Q.parallel_for(range{BLOCK_SIZE}, e, [=](id<1> i) {
		data[b * BLOCK_SIZE + i] += data[i];
	});
	if ((b + 1) < NUM_BLOCKS) {
		// Prefetch next block
		e = Q.prefetch(data + (b + 1) * BLOCK_SIZE, BLOCK_SIZE);
	}
}

Q.wait();

free(data, Q);
free(read_only_data, Q);
\end{lstlisting}

最简单的方法是调用预取(prefetch)。此函数作为处理程序或队列类的成员函数调用,并接受基指针和字节数。这可以通知运行时某些数据将在设备上使用,以便能够及时地迁移。理想情况下,应该尽早发出这些提示,这样当内核访问到数据时,数据已经驻留在设备上,从而消除之前所说的延迟。\par

DPC++提供的另一个函数是mem\_advise。这个函数允许提供,如何在内核中使用特定于设备内存的提示。我们可以指定这样一个通知,数据只在内核中读取,而不是写入。这样,系统可以意识到这个操作可以复制或复制设备上的数据,这样在内核完成后就不需要更新主机的数据。但是,传递给mem\_advise的建议是特定于设备的,所以使用此函数之前,请查阅硬件的文档。\par
















