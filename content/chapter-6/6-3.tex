USM可以以各种不同的方式分配内存,以满足不同的需求。在更详细地讨论所有方法之前,应该了解一下USM分配与C++分配的区别。\par

\hspace*{\fill} \par %插入空行
\textbf{需要知道什么?}

普通的C++程序可以以多种方式分配内存:new、malloc或分配器。不管使用哪种语法,内存分配最终都由主机操作系统中的系统分配器执行。当在C++中分配内存时,唯一需要考虑的是“需要多少内存?”和“可以分配多少内存?”但是,USM需要更多的信息才能进行分配。\par

首先,USM分配需要指定分配类型:device、host或shared。为了获得该分配所需的行为,使用正确的分配类型非常重要。每个USM必须指定一个上下文对象来进行分配,上下文表示可以在对应的设备上执行内核。可以把上下文看作是运行时存储设备状态的容器。大多数DPC++程序中,开发者可能不直接与上下文交互,而只是传递上下文。\par

不能保证USM分配可以跨不同的上下文使用——所有USM的分配、队列和内核共享同一个上下文对象。通常,可以从队列中获得上下文。设备分配还要求指定在哪个设备上分配内存,因为我们不想过度分配设备的内存(除非设备能够支持这一点——本章后续会对数据迁移时进行更多的讨论)。USM分配例程可以通过添加参数来区别于C++原生的方式。\par

\hspace*{\fill} \par %插入空行
\textbf{多重样式}

想用单一的选项来取悦每个人是不可能的,就像有些人喜欢咖啡而不是茶,或者喜欢emacs而不是vi一样。USM支持选择的多样性,并提供了几种不同类型的分配接口:C风格、C++风格和C++分配器风格。将逐个讨论,并指出相同点和不同点。\par

\hspace*{\fill} \par %插入空行
\textbf{C风格的分配方式}

第一种类型的分配函数(图6-2中列出,在随后的图6-6和6-7中中使用): malloc函数中的内存分配,函数需要返回一个void *指针(模仿C语言)。同时,必须指定要分配的总字节数,当要分配N个类型为X的对象,就必须要求总字节数为N * sizeof(X)。返回的指针为void *类型,必须将其转换为指向X类型的指针。这种方式非常简单,但由于需要进行大小计算和类型转换,因此会让代码看起来很冗长。\par

可以进一步将这种分配方式分为两类:命名函数和单函数。这两种方式的区别在于我们如何指定所需的USM分配类型。对于命名函数(malloc\_device、malloc\_host和malloc\_shared),USM分配的类型编码在函数名中,单函数malloc要求将USM分配的类型指定为一个参数。具体要使用哪种方式,还是要取决于我们的偏好。\par

这里先要提及的概念是内存对齐。malloc的每个版本都有aligned\_alloc对应项。malloc函数返回与设备默认对齐的内存,它将返回一个合法的指针和一个有效的对齐方式。但在某些情况下,我们可能更喜欢手动指定对齐的方式,使用aligned\_alloc变量指定所需的对齐方式。如果指定的方式非法,不要期望程序能正常工作!原则上的是2的幂次方。值得注意的是,在许多设备上,分配是最大对齐的,以对应硬件的特性。因此,我们可能要求分配为4字节、8字节、16字节或32字节对齐,但在实际可能会看到更大字节的对齐。\par

\hspace*{\fill} \par %插入空行
图6-2 C风格的USM分配函数
\begin{lstlisting}[caption={}]
// Named Functions
void *malloc_device(size_t size, const device &dev, const context &ctxt);
void *malloc_device(size_t size, const queue &q);
void *aligned_alloc_device(size_t alignment, size_t size,
					const device &dev, const context &ctxt);

void *aligned_alloc_device(size_t alignment, size_t size, const queue &q);

void *malloc_host(size_t size, const context &ctxt);
void *malloc_host(size_t size, const queue &q);
void *aligned_alloc_host(size_t alignment, size_t size, const context
&ctxt);
void *aligned_alloc_host(size_t alignment, size_t size, const queue &q);

void *malloc_shared(size_t size, const device &dev, const context &ctxt);
void *malloc_shared(size_t size, const queue &q);
void *aligned_alloc_shared(size_t alignment, size_t size,
							const device &dev, const context &ctxt);
void *aligned_alloc_shared(size_t alignment, size_t size, const queue &q);

// Single Function
void *malloc(size_t size, const device &dev, const context &ctxt,
			 usm::alloc kind);
void *malloc(size_t size, const queue &q, usm::alloc kind);
void *aligned_alloc(size_t alignment, size_t size,
					const device &dev, const context &ctxt,
					usm::alloc kind);
void *aligned_alloc(size_t alignment, size_t size, const queue &q,
					usm::alloc kind);
\end{lstlisting}

\hspace*{\fill} \par %插入空行
\textbf{C++风格的分配方式}

下一个USM分配函数(在图6-3中列出)与第一个非常相似。我们再次拥有分配例程的命名和单函数版本,以及默认和用户指定的对齐版本。不同之处在于,现在的函数是C++模板化函数,它分配类型为T的Count对象并返回类型为T *的指针。可以利用现代C++进行简化,不再需要以字节为单位手动计算分配的总大小,或者将返回的指针强制转换为适当的类型。这也会让代码更紧凑、不容易出错。然而,与C++中的new不同,malloc风格的接口并不为分配的对象调用构造函数——只是分配了足够的字节来适配该类型。\par

对于使用USM编写的代码来说,这种分配方式是很好的起点。对于大量使用C或C++ malloc的现有C++代码来说,C的方式是很好的起点,我们将在此基础上增加USM的使用。\par

\hspace*{\fill} \par %插入空行
图6-3 C++风格的USM分配函数
\begin{lstlisting}[caption={}]
// Named Functions
template <typename T>
T *malloc_device(size_t Count, const device &Dev, const context &Ctxt); 
template <typename T> 
T *malloc_device(size_t Count, const queue &Q); 
template <typename T>
T *aligned_alloc_device(size_t Alignment, size_t Count, const device &Dev,
						const context &Ctxt); 

template <typename T>
T *aligned_alloc_device(size_t Alignment, size_t Count, const queue &Q); 

template <typename T> T *malloc_host(size_t Count, const context &Ctxt);
template <typename T> T *malloc_host(size_t Count, const queue &Q);
template <typename T>
T *aligned_alloc_host(size_t Alignment, size_t Count, const context &Ctxt);
template <typename T>
T *aligned_alloc_host(size_t Alignment, size_t Count, const queue &Q);、

template <typename T>
T *malloc_shared(size_t Count, const device &Dev, const context &Ctxt);
template <typename T> T *malloc_shared(size_t Count, const queue &Q);
template <typename T>
T *aligned_alloc_shared(size_t Alignment, size_t Count, const device &Dev,
						const context &Ctxt);				
template <typename T>
T *aligned_alloc_shared(size_t Alignment, size_t Count, const queue &Q);

// Single Function
template <typename T>
T *malloc(size_t Count, const device &Dev, const context &Ctxt,
			usm::alloc Kind);
template <typename T> T *malloc(size_t Count, const queue &Q, usm::alloc
Kind);
template <typename T>
T *aligned_alloc(size_t Alignment, size_t Count, const device &Dev,
				const context &Ctxt, usm::alloc Kind);
template <typename T>
T *aligned_alloc(size_t Alignment, size_t Count, const queue &Q,
				usm::alloc Kind);
\end{lstlisting}

\hspace*{\fill} \par %插入空行
\textbf{C++分配器}

USM分配的最后一种风格(图6-4)使用了现代C++,这种风格基于C++的allocator接口,该接口定义了直接或间接地在容器(如std::vector)中执行内存分配的对象。如果代码大量使用容器对象,可以向用户隐藏内存分配和回收的细节,简化代码减少出现bug的机会,所以这种分配器风格最为实用。\par

\hspace*{\fill} \par %插入空行
图6-4 C++分配器风格的USM分配函数
\begin{lstlisting}[caption={}]
template <class T, usm::alloc AllocKind, size_t Alignment = 0>
class usm_allocator {
public:
	using value_type = T;
	template <typename U> struct rebind {
		typedef usm_allocator<U, AllocKind, Alignment> other;
	};

	usm_allocator() noexcept = delete;
	usm_allocator(const context &Ctxt, const device &Dev) noexcept;
	usm_allocator(const queue &Q) noexcept;
	usm_allocator(const usm_allocator &Other) noexcept;
	template <class U> 
		usm_allocator(usm_allocator<U, AllocKind, Alignment> const &) noexcept;
		
	T *allocate(size_t NumberOfElements); 
	void deallocate(T *Ptr, size_t Size); 
	
	template <
		usm::alloc AllocT = AllocKind,
		typename std::enable_if<AllocT != usm::alloc::device, int>::type = 0,
		class U, class... ArgTs>
	void construct(U *Ptr, ArgTs &&... Args); 
	
	template <
		usm::alloc AllocT = AllocKind,
		typename std::enable_if<AllocT == usm::alloc::device, int>::type = 0,
		class U, class... ArgTs>
	void construct(U *Ptr, ArgTs &&... Args); 
	
	template <
		usm::alloc AllocT = AllocKind,
		typename std::enable_if<AllocT != usm::alloc::device, int>::type = 0>
	void destroy(T *Ptr);
	
	template <
		usm::alloc AllocT = AllocKind,
		typename std::enable_if<AllocT == usm::alloc::device, int>::type = 0>
	void destroy(T *Ptr);
};
\end{lstlisting}

\hspace*{\fill} \par %插入空行
\textbf{释放内存}

无论分配什么,最终都必须释放。USM定义了一个方法来释放malloc或aligned\_malloc函数分配的内存。这个方法还将分配内存的上下文作为一个参数(可以用队列替换上下文)。如果内存是用C++的allocator对象分配,也应该使用该对象来释放内存。\par

\hspace*{\fill} \par %插入空行
图6-5 三种配置方式
\begin{lstlisting}[caption={}]
constexpr int N = 42;

queue Q;

// Allocate N floats

// C-style
float *f1 = static_cast<float*>(malloc_shared(N*sizeof(float),Q));

// C++-style
float *f2 = malloc_shared<float>(N, Q);

// C++-allocator-style
usm_allocator<float, usm::alloc::shared> alloc(Q);
float *f3 = alloc.allocate(N);

// Free our allocations
free(f1, Q.get_context());
free(f2, Q);
alloc.deallocate(f3, N);
\end{lstlisting}

\hspace*{\fill} \par %插入空行
\textbf{内存分配示例}

图6-5中,展示了如何使用三种方式进行分配,我们分配N个单精度浮点数作为共享分配内存。第一个分配f1使用C风格的void *返回malloc例程。对于这种分配,我们显式地传递从队列中获得的设备和上下文。\par

必须将结果强制转换为float*类型。第二个分配f2做了同样的事情,但是使用了C++风格的malloc模板。因为我们将元素的类型float传递给分配示例,所以只需要指定分配多少个float即可,而不需要对结果进行强制转换。还可以使用队列,而不是设备和上下文的形式,完成一段非常简单和紧凑的代码。第三个分配f3使用了USM C++的allocator类,实例化了allocator对象,然后使用该对象执行分配。最后,展示了如何正确地释放分配的内存。\par












































