第12章引入了一组富有表现力的查询,使主机能够在运行时提取有关设备的信息。这些查询允许针对特定设备调优工作组大小等运行时参数,并允许将实现不同算法的不同内核分派到不同类型的设备。\par

未来的版本预计将使用编译时查询来扩充这些运行时查询,允许基于实现是否理解供应商扩展而对代码进行专一化。图EP-5显示了如何使用预处理器来检测编译器是否支持特定的供应商扩展\par

\hspace*{\fill} \par %插入空行
图EP-5 使用\#ifdef检查Intel子组扩展编译器支持
\begin{lstlisting}[caption={}]
#ifdef SYCL_EXT_INTEL_SUB_GROUPS
sycl::ext::intel::sub_group sg = it.get_sub_group();
#endif
\end{lstlisting}

我们还计划引入编译时查询,使内核能够根据目标设备的属性(我们称之为方面)进行专门化(例如,设备类型、对特定扩展的支持、工作组本地内存的大小、编译器选择的子组大小)。这些方面代表了一种特殊的常量表达式,目前在C++中不存在——它们在编译宿主代码时不一定是constexpr,但在知道目标设备时就变成了constexpr。用于公开设备constexpr的机制仍在设计中。我们希望它以SYCL 2020临时引入的专用常量化特性为基础,并且形式和行为上类似于图EP-6中所示。\par

\hspace*{\fill} \par %插入空行
图EP-6 内核编译时基于设备方面特化内核代码
\begin{lstlisting}[caption={}]
h.parallel_for(..., [=](item<1> it) {
	if devconstexpr (this_device().has<aspect::cpu>()) {
		/* Code specialized for CPUs */
	}
	else if devconstexpr (this_device().has<aspect::gpu>()) {
		/* Code specialized for GPUs */
	}
});
\end{lstlisting}






















































