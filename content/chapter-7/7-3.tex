In this chapter, we have learned about buffers and accessors. Buffers are an abstraction of data that hides the underlying details of memory management from the programmer. They do this in order to provide a simpler, higher-level abstraction. We went through several examples that showed us the different ways to construct buffers as well as the different optional properties that can be specified to alter their behavior. We learned how to initialize a buffer with data from host memory as well as how to write data back to host memory when we are done with a buffer.\par

Since we should not access buffers directly, we learned how to access the data in a buffer by using accessor objects. We learned the difference between device accessors and host accessors. We discussed the different access modes and targets and how they inform the runtime how and where an accessor will be used by the program. We showed the simplest way to use accessors using the default access modes and targets, and we learned how to distinguish between a placeholder accessor and one that is not. We then saw how to further optimize the example program by giving the runtime more information about our accessor usage by adding access tags to our accessor declarations. Finally, we covered many of the different ways that accessors can be used in a program.\par

In the next chapter, we will learn in greater detail how the runtime can use the information we give it through accessors to schedule the execution of different kernels. We will also see how this information informs the runtime about when and how the data in buffers needs to be copied between the host and a device. We will learn how we can explicitly control data movement involving buffers—and USM allocations too.\par

\newpage